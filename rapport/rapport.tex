\documentclass[a4,12pt]{article}

\usepackage[francais]{babel}
\usepackage[utf8]{inputenc}
\usepackage[T1]{fontenc}
\usepackage[babel=true]{csquotes}
\usepackage{amsmath}
\usepackage{amssymb}
\usepackage{float}
\usepackage{graphicx}
\usepackage{hyperref}
\setlength\parindent{20pt}

\begin{document}
\begin{titlepage}
  \title{Exploit de la vulnérabilité CVE-2016-1494\
    Contrefaçon de signature dans le package python-rsa
  }
  \author{Valérian Baillet, Matthias Beaupère}
  \date{}
\end{titlepage}

\maketitle

\section{Introduction}

La vulnérabilité CVE-2016-1494 a été soumise par Filippo Valsorda le 5 janvier 2016. Il s'agit d'une possibilité de falsifier des signatures à exposant faible dans la module rsa de python. Ce faille est présente dans les versions antérieures à 3.3.

\section{Les signatures RSA}

Pour signer un message en RSA, on joint ce message d'un hash du message, chiffré avec la clé privé de l'expéditeur comme ceci :
$$
m^e mod N = s
$$
Avec $m$ le hash du message à signer, $e$ l'exposant et $N$ le modulo de la clé privé de l'expéditeur. On obtient $s$ la signature.\\

Lorsque le correspondant reçoit la signature $s$, il effectue l'opération suivante :
$$
s^d mod N = m
$$
Avec $d$ la clé publique de l'expéditeur.

Il calcule ensuite le hash du message et le compare à la signature déchiffrée $m$.







\end{document}
