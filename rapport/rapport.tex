\documentclass[a4,12pt]{article}

\usepackage[francais]{babel}
\usepackage[utf8]{inputenc}
\usepackage[T1]{fontenc}
\usepackage[babel=true]{csquotes}
\usepackage{amsmath}
\usepackage{amssymb}
\usepackage{float}
\usepackage{graphicx}
\usepackage{hyperref}
\setlength\parindent{20pt}

\begin{document}
\begin{titlepage}
  \title{Exploit de la vulnérabilité CVE-2016-1494\
    Contrefaçon de signature dans le package python-rsa
  }
  \author{Valérian Baillet, Matthias Beaupère}
  \date{}
\end{titlepage}

\maketitle

\section{Introduction}

La vulnérabilité CVE-2016-1494 a été soumise par Filippo Valsorda le 5 janvier 2016. Il s'agit d'une possibilité de falsifier des signatures dans la module rsa de python. Ce faille est présente dans les versions antérieures à 3.3.
Nous étudions ici cette faille...


\end{document}
